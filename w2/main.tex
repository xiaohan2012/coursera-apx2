\documentclass{article}
\usepackage{graphicx}
\usepackage{xspace}
\usepackage{amsmath}

\begin{document}

\title{Approximation Algorithms Part II: week 2 assignment}
\author{Anonymous}

\maketitle

\newcommand{\abs}[1]{|#1|}
\newcommand{\val}{\text{val}}

\section*{Primal-dual algorithm for Shortest Path problem}

\paragraph{Q1. }

\begin{tabular}{crr}
  maximize  & $\sum_{S \in \mathcal{S}} y_S$ & \\
  s.t.      & $\sum_{S \in \mathcal{S}: e \in S} y_S \le w(e)$ & $\forall e \in E$ \\
            & $y_S \ge 0$ & $\forall S \in \mathcal{S}$
\end{tabular}

\paragraph{Q2. }

At most $|V|-1$ iterations.

\paragraph{Q3. }

After each iteration, an edge is added and the size of $C$ increases by 1.
After $|V|-1$ iterations, $C$ contains all $V$.
So there is at least one path from $s$ to $t$ when the algorithm terminates. 

\paragraph{Proof of Lemma 1}

We want to show after $i$ iterations, $\abs{F_i}=i$. 
Initially, $\abs{F_0}=0$ because $C$ contains only $s$. 
After the 1st iteration, $\abs{F_1}=1$ because step 3 adds one edge.
Assuming after iteration $i-1$, $\abs{F_{t-1}}=i-1$.
Then after iteration $i$, $\abs{F_t}=\abs{F_{t-1}}+1=t$ because of step 3.
And this concludes our proof.

\paragraph{Q4. }

$\val(y^{*}) \le \val(P^*)$ due to weak complementary slackness.

\paragraph{Q5. }

At any iteration, suppose an edge $e^{′}$ is added to $F$. And nods in $e^{'}$ are included into $C$. Obviously, $e$ is not crossing $C$ and $V \setminus C$. 
In the next step, we only increase $y_C$. Therefore, the constraint in dual related to $w(e^{'})$ is never broken. Therefore, the solution $y$ is feasible for the dual. 

\paragraph{Q6. }

We have $\val(y) \le \val(P^*)$.

\paragraph{Q7. }

$w(e) = \sum_{S \in \mathcal{S}: e \in \delta(S)} y_S$. 

\paragraph{Q8. }

$\sum_{e \in P} w(e) = \sum_{e \in P} \sum_{S \in \mathcal{S}: e \in \delta(S)} y_S$ 

\paragraph{Q9. }

$\sum_{e \in P} \sum_{S \in \mathcal{S}: e \in \delta(S)} y_S = \sum_{S \in \mathcal{S}} y_S \sum_{e \in P \cap \delta(S)} = \sum_{S \in \mathcal{S}} y_S \abs{P \cap \delta(S)}$

\paragraph{Q10. }

If $|P\cap \delta(S)|>=2$, it means there is a cycle in some $S^{'}$ after $S$. This contradicts with the fact that $S^{'}$ is a tree (Lemma 1). 

\paragraph{Q11. }

The primal-dual algorithm is optimal. 

\paragraph{Q12. }

Intuitively, pruning is needed to remove unnecessary edges. It's used in question 10. 

\paragraph{Q13. }

The forward direction: if $(s, i)$ is the first edge added to $F$, it means $(s, i)$ has the minimum $w(s, j)$ among all neighbors $j$ of $s$.
In Dijkstra algorithm, $i$ is not in $D$ and minimizes $d(i)$. So it's selected.

The reverse direction, if $i$ is selected, it means $d(i)$ is minimal among all neighbors of $s$. And $d(i)$ is initialized to be $w(s, i)$. Therefore, $(s, i)$ is chosen first by the primal-dual algorithm. 

\paragraph{Q14. }

The vertex $j$ with minimum $l(j)$ will be added to $C_0$. 

\paragraph{Q15. }

$\{i \in \text{neighbor}(j): i \not\in S^{'}\}$, i.e. $j$'s neighbors that are not in $S^{'}$ 

\paragraph{Q16. }

For $k \not\in S^{'}, l(k) = l(j) + w(k, j)$. 

\paragraph{Q17. }

The same order! 

\paragraph{Q18. }

$O(n^2)$ if you consider a complete graph. 

\end{document}
